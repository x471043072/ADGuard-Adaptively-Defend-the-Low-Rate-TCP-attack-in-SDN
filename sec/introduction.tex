\section{Introduction} 
DoS (Denial of Service) attacks are great threats to Internet security. Recently, a 1.35 Tbps DoS attack took the world's leading software development platform GitHub offline for several minutes~\cite{b19}. Among all kinds of DoS attacks, the low-rate TCP attack~\cite{b20} is essentially the most efficient in terms of causing damage to benign TCP flows. Instead of directly sending huge amount of traffic, it generates periodically pulsing flows to cause continuous retransmission of bening TCP flows, which result in significant throughput degradation with low-rate attack flows. Due to the low rate, such attack is more difficult to be detected compared to flooding-based attacks. Moreover, the attack can be launched in a distributed mode, which further increases the stealthiness and the difficultly of effectively throttling the attack flows.


%, in which the attacker sends periodical traffic to overload the bandwidth of a router and leads to packet losses in a link. By synchronizing the attack period to the RTO duration, the low-rate TCP attack induces  of the TCP flows ly. 

%Detecting the low-rate TCP attack is much more difficult than detecting traditional DoS attacks because of the low average rate. In addition, any router forwarding the TCP flows can be the target of the attack. Moreover, The attack can be launched in a distributed mode, therefore, the attack sources are hard to be located. Once a TCP flow becomes a victim of the low-rate TCP attack, the throughput of the flow will become very low. Hence, the protocols based on TCP are influenced seriously, e.g. HTTP, FTP, and BGP.
 
To effectively defend against the attack, several defenses have been proposed. Most defenses~\cite{b1,b4, b6, b7, b22} identify the attack flows by applying digital signal processing (DSP) techniques to extract frequency-domain characteristics. Such characteristics can only be well depicted when a proper sampling period of collecting flow statistics is set. However, existing defenses use a fixed sampling period, which may result in failing to detect attack flows with short periods. Besides, they do not consider how to throttle the identified flows. A few defenses adopt passive methods to counter the low-rate TCP attack without identifying attack flows, such as randoming RTO  in the hosts \cite{b17} and modifying the fairness mechanism of dropping packets in the switches \cite{b8}. Such passive defenses are not easy to be deployed in practice due to the modification of the network protocol stack in hosts or hardwares in switches. They also can not effectively throttle the attack flows.

%They require proper sampling period of  to 

%analyzed the traffic by exploring \emph{} (DSP) in time domain and frequency domain . By examing the , \cite{b3} detects the low-rate TCP attack quickly. Other schemes also mitigate the low-rate TCP attack with different mechanisms.  randomized the TCP RTO to counter the low-rate TCP attack, which needs to be deployed in the end host and modifies the default behavior of TCP.  modifies fairness of drop packets in the switches and prevents the low-rate TCP attack by dropping packets. Unfortunately, none of the above defense schemes could adaptively detect the low-rate TCP attack and make deployment easier.

Recently, Software-Defined Networking (SDN) has been emerged as a promising network paradigm. Due to its centralized control and flexible programmability, it shows great benefits on defending DoS attacks. Several SDN-based approaches~\cite{b9, b16, b11, b23, b24} have been proposed to detect and throttle various DoS attacks. However, existing defenses focus on defending against flooding-based DoS. They fail to consider on how to detect and mitigate the low-rate TCP attack. Now that the attack is low-rate and has great differences on the characteristics of the attack flows compared to the flooding attacks.

%SDN provides new methods to defend against DDoS attacks. Flow information collected from the SDN switches and the global view of the network provided by the SDN controller can be utilized to analyze DDoS attacks. With the support of SDN, the defense system could detect the attack flows, even locate the attack sources. Recently several SDN-based approaches have been proposed to detect and mitigate DDoS \cite{b9}, \cite{b16}, \cite{b11}, \cite{b23}, \cite{b24}. With these approaches, various DDoS attacks are effectively and efficiently detected. However, they fail to detect and defend low-rate TCP attack. The target of low-rate TCP attack is the link of TCP flows' paths, and the average rate of low-rate TCP attack is reduced compared with the other DDoS, hence the low-rate TCP attack is harder to be detected than the flooding style attack.

In this paper, we propose \TheName{}, which is a lightweight SDN-based defense system to accurately identify and effectively throttle the attack flows.  \TheName{} consists of three modules: \emph{monitor}, \emph{locator}, and \emph{mitigator}. The monitor module detects the low-rate TCP attack by installing crafed flow rules to check the TCP throughput of each ports of switches. Here, we do not monitor the throughput of each flow as it will consume massive bandwidth of controllers considering so many flows in the network. When significant throughput degradation is monitored, the locator module will be activated to identify attack flows and locate their sources. The key to identify attack flows is to judge whether there is periodicity for flows, since attack flows have remarkable period. We design an adaptive Fast Fourier Transfrom (FFT) method to extract different period of attack flows. Moreover, the locator module will locate the sources of the attack flows based on the flow paths. Afterwards, the locator module installs mitigation rules in the ingress switches of the sources to throttle the attack flows. 

%an adaptive approach to detect and mitigate the low-rate TCP attack by analyzing the periodicity of the low-rate TCP attack. If the throughput of TCP flows is reduced in a switch, we monitor all the active ports of the switch. The key to detecting the low-rate TCP attack is to determine the periodicity of the rate data in our approach. To determine the periodicity, we collect the time-domain signal by taking samples of packet arriving rate with decreasing sampling period, and transform it into the frequency domain by FFT (\emph{Fast Fourier Transfrom}) for the constant period of low-rate TCP attack, as the period of the samples can be obtained by analyzing FFT. And then, we compare the statistics of the port detected to be attacked by the low-rate TCP attack with the statistics of the flow rules related the port to detect the attack flows. After the attack flows are detected, we identify the attackers by analyzing the match fields of flow rules forwarding attack traffic and isolate them by installing flow rules to drop packets from the attackers. Our detection minimizes the overhead of the controller by adaptive sampling period. The influence of the low-rate TCP attack is minimized by mitigating the attack in the access switches. Notice that though the low-rate TCP attack is distributed, the attack can be detected and mitigated in our system.

We conduct experiments with a real SDN testbed consisting of hardware switches to evalute the effectiveness of \TheName{}. The results show that 
more than 90\% attack flows with various period can be detected by our system. Moreover, \TheName{} can effectively protect the throughput of TCP flows from the attack flows while introduce a negligible overhead.
%the period of the low-rate TCP attack can be inferred during the period are increased from 0.1 s to 1 s in an acceptable error, meanwhile, the attack flows can be detected in an appropriate sampling period. Based on the analysis of more than 1,000 simulations, more than 90\% attack flows can be detected with the increasing period from 0.8s to 1.5s. The throughput with the protection of our system is far higher than that of without protection, meanwhile, it introduces additional 14\% CPU usage at most. 

To summarize, we make the following contributions:

\begin{itemize}

\item We propose a defense system called \TheName{}, which can effectively defend low-rate TCP attack without any modifications in the switches or SDN protocols.  
\item We develop algorithms that can accurately identify the malicious flows of the low-rate TCP attack.
\item We implement \TheName{} in the Floodlight controller with a real hardware testbed, and conducts experiments to evaluate the effectiveness.

\end{itemize}
