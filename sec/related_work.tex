\section{Related Work}

\noindent \textbf{Defenses against the low-rate TCP attack.} Several methods have been proposed to detect and mitigate the low-rate TCP attack in traditional IP networks. In the study\cite{b4}, the authors proposed a mechanism to dynamically detect and defend against the low-rate TCP attack. However, the mechanism fails to consider mitigating the malicious flows. In addition, it sets a fixed sampling period, which means the attack flows whose periods are smaller than the sampling periods cannot be identified. 
%In the study\cite{b5}, the author proposed four alternatives to defend the low-rate DoS attack against application servers. The  approaches randomize the server operation to rearrange the positions in the incoming queues of applications for eliminating possible vulnerabilities due to predictable behaviors. 
Other methods adopts frequency based methods, including the sum of Low-Frequency Power spectrum (SLFP) \cite{b6} and cumulative amplitude spectrum (CAS) \cite{b7}. Shrew Attack Protection (SAP) \cite{b8} is a port-based approach by changing the fairness mechanism of dropping packets. \cite{b17} introduces randomization on TCP RTO to defense the low-rate TCP attack. 


\noindent \textbf{Defenses against DoS attacks in SDN.} Bohatei~\cite{b9} is a flexible defense to defend many types of known DDoS, e.g., SYN flood, DNS amplification, UDP flood. %In [14], with the entropy-based method, SVM classifier, and mitigation agent in SDN, the network is protected from DDoS attacks. 
SPIFFY \cite{b16} introduces SDN-based defense to defend link-flooding attacks. 
%ProDefense \cite{b11} solves the problem that traffic threshold for DDoS detection vary for existing SDN-based solutions and utilizes a distributed controller platform for load balancing and failure tolerance improvement. 
Other defenses \cite{b10, b12, b13, b15, b18} are proposed to defend SDN DoS attacks that exhaust control-data plane bandwidth, computational resources, and flow table. However, above existing defenses fail to consider on defending the low-rate TCP attack. 


